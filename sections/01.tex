وَلَيْسَ الْبِرُّ بِأَن تَأْتُوا الْبُيُوتَ مِن ظُهُورِهَا وَلَٰكِنَّ الْبِرَّ مَنِ اتَّقَىٰ وَأْتُوا الْبُيُوتَ مِنْ
أَبْوَابِهَا وَاتَّقُوا اللَّهَ لَعَلَّكُمْ تُفْلِحُونَ. سوره البقره -۲- آیه ۱۸۹.

و نیکی به آن نیست که از پشت بام خانه‌ها به خانه‌ها درآئید، ولکن نیکویی
روش متقیان است، و از درهای خانه‌ها به خانه‌ها درآئید، و از خدا بهراسید،
شاید رستگار شوید.

\subsection{بیان}

در این آیه مبارکه خدای متعال تذکر می‌دهد که هر مقصدی را از راهش بروید،
و هر مطلوبی را از درش وارد شوید، زیرا بیراهه گمراهی است، و طالب را به
مقصد نمی‌رساند، و از غیر در وارد شدن کار خائنان و دزدان است.

یکی از چیزهایی که مطلوب هر عاقلی است دانش است، در احادیثی رسول خدا(صلی
الله علیه و آله و سلم) می‌فرماید: مرکز دانش منم، و راه رسیدن به آن
خاندان من هستند، اکنون برخی از آن احادیث

\subsection{حدیث اول}

عن ابن عباس قال: قال رسول الله صلی الله علیه (وآله) وسلم: أَنَا مَدِينَةُ
اَلْعِلْمِ وَ عَلِيٌّ بَابُهَا فَمَنْ أَرَادَ اَلْمَدِينَةَ فَلْيَأْتِ اَلْبَابَه.

عبدالله بن عباس گفت: رسول خدا(صلی الله علیه و آله و سلم) فرمود: منم
شهر علم و علی باب آن، پس هر کس علم می‌خواهد، از درش آید.

این حدیث را روایت کرده:

\begin{enumerate}
\item
  ابوبکر بغدادی

\item
  ابوالقاسم سهمی

\item
  ابن المَغازِلی

\end{enumerate}
