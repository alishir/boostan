سپاس خدا راست که انسان آفرید، و برای تکامل عرفان به این خاکدانش کشید،
و درود بی‌پایان، بر سرور پیامبران و خاندان پاکش که مکتب عالی آموزش
انسانی را گشودند، و جهان را با نشر تعالیم و معارف الاهی بوستان معرفت
نمودند.

این کتاب قطره‌ای از دریای بی‌کران فضائل خاندان نبوّت است، به اهل ایمان و
محبّت اهداء می‌شود، و امید از خدای منان که عنایت فرماید، و ما را در دو
جهان با آنان محشور بدارد، اکنون سخنی چند به عنوان پیش‌داشت مذکور
می‌داریم.

\subsection*{سخن اول}

اصل هر فضیلتی دانش است، و هر فضیلتی در ظرفِ جهل وارونه گردد، شجاعِ جاهل،
مُتَهَوِّر است، خود و دیگران را بیهوده به هلاکت اندازد، سخیِّ جاهل، مُسرِف است،
مال را بی‌جا تلف نماید، حَلیمِ جاهل مُنْظَلِم است، بی‌جهت زیر بار ستم رود،
عابد جاهل، مُبْدِع است، عباداتی بر خلاف سُنّت آورد، زاهد جاهل، مُهْمل کار
است، بسا زهدها کند که ترک وظیفه باشد، واعظ جاهل مُفْسِدْ است، بسا سخنان
گوید که مردم را مُنْحَرِف نماید، قاضی جاهل ظالم است، بسا داوری‌ها کند که حق
مظلوم ضایع شود، شرمسار جاهل بی‌جا شرم نماید، و خود را از مزایایی محروم
سازد، راستگوی جاهل فتنه انگیز است، بسا راست‌ها گوید و فتنه‌ها از آن
برخیزد، ووو.

پس علم است که هر رذیلتی را جلوگیر، و هر فضیلتی را به جا استوار نماید،
و از این روی دانسته شود که مِلاکِ برتری هر فردی، بر فردی و هر اجتماعی بر
اجتماعی دانش است.

ولی به شرط تَقوی، و دانشمندِ بی‌تقوی از نادان پست‌تر است، زیرا برای خود و
جامعه زیان‌‌بار است، ولی نادان خود را از منافعی محروم ساخته.

خدای متعال گرچه دانش و دانشمند را در آیاتی ستوده، ولی گرامی بودن نزد
خودش را به شرط تقوی گوشزد نموده، إِنَّ أَكْرَمَكُمْ عِنْدَ اللَّهِ أَتْقَاكُمْ: همانا
گرامی‌ترین شما نزد خدا پرهیزکارترین شماست.

خلاصه سخن دو اصل مُثْبَت و مَنفی برای تکامل بشری لازم است: علم و تقوی، و
هر یک به تنهایی ثمر بخش نیست.

\subsection*{سخن دوم}

در این کتاب احادیثی از رسول اکرم(صَلَّی اللهُ عَلَیْه و آله و سلّم) در چهل بخش
درباره علم خاندان نُبُوَّت(صَلَواتُ الله علیهم) آورده‌ام، و بخش چهلم، سخنان
برخی از صحابه و تابعین و دانشمندان است، و آن‌ها گرچه وانمود سخنان پیمبر
است، لکن گواهی است، که امت همگی بفضل علمی آنان اعتراف دارند، و در این
باره کسی بر سر انکار نیست، و به شهادت احادیثی از رسول خدا(صلی الله
علیه و آله و سلم) فضائل آنان بی‌نهایت است، برخی از آن احادیث:

عَنِ اِبْنِ عَبَّاسٍ رَضِیَ اللهُ عَنْهُ قَالَ: قَالَ رَسُولُ اَللَّهِ صَلَّى اَللَّهُ عَلَيْهِ وَ آلِهِ: لَوْ أَنَّ
اَلْغِيَاضَ أَقْلاَمٌ وَ اَلْبَحْرَ مِدَادٌ وَ اَلْجِنَّ حُسَّابٌ وَ اَلْإِنْسَ كُتَّابٌ مَا أَحْصَوْا فَضَائِلَ عَلِيِّ
بْنِ أَبِي طَالِبٍ عَلَیْهِ السَّلامُ.

ابن عباس(رَضِیَ اللهُ عَنْهُ) گفت: رسول خدا(صَلَّى اَللَّهُ عَلَيْهِ وَ آلِهِ وَ سَلَّمَ) فرمود:
اگر درختان قلم، و دریاها مداد، و پریان حسابگر، و آدمیان نویسنده باشند،
فضائل علی بن ابی طالب را نتوان به حساب آورند.

عَنْ اَمیرِالْمُوْمِنینَ عَلَیْهِ السَّلامُ قالَ: قَالَ رَسُولُ اَللَّهِ صَلَّى اَللَّهُ عَلَيْهِ وَ آلِهِ: إِنَّ
اَللَّهَ تَعَالَى جَعَلَ لِأَخِي عَلِيٍّ فَضَائِلَ لاَ تُحْصَى كَثْرَةً فَمَنْ ذَكَرَ فَضِيلَةً مِنْ فَضَائِلِهِ
مُقِرّاً بِهَا غَفَرَ اَللَّهُ لَهُ مَا تَقَدَّمَ مِنْ ذَنْبِهِ وَ مَا تَأَخَّرَ وَ مَنْ كَتَبَ فَضِيلَةً مِنْ
فَضَائِلِهِ لَمْ تَزَلِ اَلْمَلاَئِكَةُ تَسْتَغْفِرُ لَهُ مَا بَقِيَ لِتِلْكَ اَلْكِتَابَةِ رَسْمٌ وَ مَنِ اِسْتَمَعَ
إِلَى فَضِيلَةٍ مِنْ فَضَائِلِهِ غَفَرَ اَللَّهُ لَهُ اَلذُّنُوبَ اَلَّتِي اِكْتَسَبَهَا بِالاِسْتِمَاعِ وَ مَنْ نَظَرَ
إِلَى كِتَابٍ مِنْ فَضَائِلِهِ غَفَرَ اَللَّهُ لَهُ اَلذُّنُوبَ اَلَّتِي اِكْتَسَبَهَا بِالنَّظَرِ ثُمَّ قَالَ اَلنَّظَرُ
إِلَى أَخِي عَلِيِّ بْنِ أَبِي طَالِبٍ عِبَادَةٌ وَ ذِكْرُهُ عِبَادَةٌ وَ لاَ يَقْبَلُ اَللَّهُ إِيمَانَ عَبْدٍ
إِلاَّ بِوَلاَيَتِهِ وَ اَلْبَرَاءَةِ مِنْ أَعْدَائِهِ.

امیرالمومنین(علیه السلام) گفت: رسول خدا(صَلَّى اَللَّهُ عَلَيْهِ وَ آلِهِ وَ سَلَّمَ)
فرمود: خدا برای برادرم علی فضائِلی قرار داده که از بسیاری به حساب
نیاید، پس هر کس فضیلتی از فضائل او را یاد کند و به آن اعتراف داشته
باشد، خدا گناهان گذشته و آینده‌اش را بیامرزد.

و هر کس فضیلتی از فضائل او را بنویسد تا وقتی از آن نوشته اثری باقی
است، فرشتگان همواره برای او از خدا آمرزش می‌طلبند.

و هر کس فضیلتی از فضائل علی را به گوش فراگیرد، خدا گناهانی را که با
شنیدن مرتکب شده بخشش نماید.

و هر کس در کتابی از فضائل علی بنگرد، خدا گناهانی را که با دیدن مرتکب
شده آمرزش نماید.


سپس رسول خدا(صَلَّى اَللَّهُ عَلَيْهِ وَ آلِهِ وَ سَلَّمَ) فرمود: نظر کردن به برادرم علی
بن ابی طالب عبادت است، و ذکر علی عبادت است، و خدا ایمان بنده‌ای را
نپذیرد جز به ولایت علی و بیزاری از دشمنان علی.

این دو حدیث را روایت کرده:

\begin{enumerate}
\item

حافظ اَبوالْمُؤَیَّد مُوفَّق بن اَحمد بن محمد بَکْری مَکّی حَنفی مَذْهب، معروف به خطیب
خوارَزمی و اَخْطَبِ خوارزم که در سال ۵۶۸ هجری قمری بدرود جهان گفته، در آغاز
کتابش به نام (مَناقب خوارزمی).

\item

اَبوالمَجامِع ابراهیم بن محمد بن الْمُؤَیَّد بن حَمّویَه جُوَیْنی خراسانی حنفی مذهب
معروف به حَمُّویی که در سال ۷۳۰ بدرود جهان گفته، در آغاز کتابش (فَرائِد
السِّمْطَیْنِ).

\item

 سِبْطِ اِبْنِ‌ الْجَوْزی‌‌، یوسف‌ بن‌ قُزُغْلى‌ بن عبدالله بغدادی، حنفی مذهب، که در سال
 ۶۵۴ بدرود جهان گفته، در آغاز کتاب (تَذْکِرَةُ الْخَواصّ) صفحه ۱۳، حدیث اول را
 آورده.

\item
  
ابو عبدالله محمد بن یوسف محمد نَوْفَلی كَنْجی شافعی مذهب كه در سال ۶۵۸ در
مسجد جامع شام بدست گروهی آشوب‌گر مقتول گشته، در كتاب (كِفایَةُ الطّالِب) باب
شصت و دوم، صفحه ۱۲۳ و ۱۲۴.

\end{enumerate}
  

\subsection*{سخن سوم}

در این تألیف از کتاب‌های شیعه سخنی نیاوردم، و از دانشمندان شیعه سخنی
نقل ننمودم، تا شود که برای اهل سُنَّت رَغْبتی حاصل آید، و بی‌خبران بدانند که
رِوایات فَضائل و مَناقب به شیعه اختصاصی ندارد.

اگر گوئی: احادیث هر یک از دو گروه در نزد گروه دیگر بی‌اعتبار است، و در
مقام استدلال به روایات یکدیگر استناد نمی‌نمایند، پس چگونه کتاب‌های اهل
سنت را مدرک این کتاب قرار دادی؟

پاسخ: نسبت احادیث شیعه و احادیث اهل سنت به اصطلاح دانشمندان فن منطق
(تَباین جُزئی است) یعنی: برخی از احادیث شیعه نزد اهل سنت اعتبار ندارد، و
برخی از احادیث اهل سنت نزد شیعه اعتبار ندارد، و برخی از احادیث نزد هر
دو گروه اعتبار دارد، احادیث فضائل خاندان نُبوّت (سَلامُ اللهِ عَلَیْهِم) بخصوص
فضائل علمی آنان، از گونه سوم است، هر دو گروه بر نقل و قبول آن‌ها توافق
دارند.

و از این جهت شخص آگاه از فن حدیث متوجه شود که صدور آن‌ها از رسول
اکرما(صَلَّى اَللَّهُ عَلَيْهِ وَ آلِهِ وَ سَلَّمَ) قطعی است، وگرنه تَوافُق دانشمندان حدیث
بر آن‌ها نبود، و در کتاب‌های هر دو گروه روایت نمی‌شد.

پس احادیث این کتاب که در فضائل علمی خاندان نبوت(صَلواتُ اللهِ عَلَیْهِم) است،
از مَدارک اهل سنت نقل شده، ولی در کتاب‌های دانشمندان شیعه، عَین آن‌ها یا
مانند آن‌ها نیز نقل شده است، گرچه برخی در سند برخی سخنی داشته‌اند، ولی
با در نظر گرفتن جهات و خصوصیات این احادیث در کتاب‌های شیعه و اهل سنت
برای آگاه از فُنُونِ حدیث، یَقین به صحت آن‌ها حاصل آید.

\subsection*{سخن چهارم}

در سند احادیث بحثی نکردم، و آن را بر عهده دانشمندان مُطَّلِع نهادم، ولی
بایستی مُتَذَّکِر شد که دانشمندان فَنِّ حدیث، احادیث را به اقسامی تقسیم
نموده‌اند، و مجموع آن اقسام (با تَفاوت و اختلافی که در آن‌ها و در حکم
آن‌هاست) دو گونه است: مُعْتَبَر و غیر معتبر.

و برخی از اهل حدیث هر حدیثی را از جنبه سند و راوی ملاحظه کرده‌اند، و
طبق شناخت و عدم شناخت،‌ و درستی و نادرستی راویان حکم کرده‌اند که حدیث
معتبر است یا معتبر نیست.

ولی برای دانشمند مُحَقِّق که می‌خواهد حدیث را مَدرَکِ اعتقاد یا عمل قرار دهد
این مقدار کافی نیست، بلکه مُخِلّ و مُضِرّ است، زیرا بسیار حدیثی از نظر سند
معتبر نیست، ولی سندی یا سندهای معتبر دیگری دارد، یا آنکه عقل یا قرآن
یا حدیث صحیحی بر صحت و اعتبارش دلالت می‌کند.

پس دانشمند مُحَقِّق نمی‌تواند تا حدیثی را در کتابی دید سندش درست نیست، حکم
کند که آن حدیث نادرست است، چنانچه برخی مانند ذَهَبی و اِبْنُ الْجَوْزی چنین
کرده‌اند.

بلکه بایستی تحقیق کرد، چنانچه برخی مانند سِبْطِ اِبْنُ الْجَوْزی و عَسْقَلانی و
سُیُوطی تحقیق کرده‌اند، و خطای نامحققان را در بسیاری از احادیث متذکر
شده‌اند.

و نیز در احادیث بسیاری رسول خدا(صلی الله علیه و آله و سلّم) و ائمه
اطهار فرموده‌اند: آنچه از ما روایت شده رَد نکنید و آن را دروغ مپندارید،
زیرا ممکن است از ما باشد و عقل شما آن را تَحَمُّل ننماید، در این صورت حَقی
را رد کرده‌اید، و صِدقی را تکذیب نموده‌اید.

پس اگر راهی برای معتبر دانستن حدیثی در پیش نبود، باید آن را واگذاشت،
نه مَدْرَکِ اعتقاد و عمل قرار داد، و نه تکذیبش نمود.

\subsection*{سخن پنجم}

رسول خدا(صلی الله علیه و آله و سلم) بسیاری از سخنان و بیانات را مُکَرَّر،
با تَعبیرات مُخْتَلف، در زمان‌ها و مکان‌های مُتَعَدِّد، برای یاران فرموده، و نقل
آنان برای دیگران تا به دست محدثان رسیده و در کتاب‌ها ثبت گردیده نیز
همین گونه بوده، پس از این جهت عِبارات و کَلِمات یک حدیث در کتاب‌ها مختلف
مُفَصَّل و مُخْتَصَر دیده می‌شود.

در این کتاب هر حدیثی را که از کتاب‌های مُتعدد نقل کرده‌ام، اول کتابی که
نامبر شده، آن حدیث از آن کتاب است، و در دیگر کتاب‌های نامبر شده می‌شود
آن حدیث مختصر یا مفصل یا اختلاف و تفاوتی در کلماتش باشد.

و در نزد محدثان معمول است که اختلاف کلمات حدیث را متذکر می‌شوند، ولی در
این کتاب رِعایت این جهت را لازم ندیدم.

\subsection*{سخن ششم}

خاندان نُبُوَّت(صلوات الله علیهم) چهارده تن هستند که نزد شیعه چهارده معصوم
گفته می‌شوند، ولی در حقیقت یکی هستند، و وحدت حقیقی دارند، یک نورند، هر
فضیلتی برای هر کدام ثابت باشد، برای همه آنان ثابت است، گرچه از برخی
بروز نداشته باشد، و اختلاف و تفاوت آنان از جَنْبِهِ بَشَری، و اِرْتِباطاتِ خَلْقی
است.

از این جهت هر بخشی از چهل بخش کتاب را درباره خاندان نبوت عنوان کردم،
گرچه بسیاری از احادیث مذکوره درباره امیرالمؤمنین علیه السلام است.

\subsection*{سخن هفتم}

مُعْتَقِدان نبوت محمد بن عبدالله(صلی الله علیه و آله و سلم) یک اُمَّتَنْد، زیرا
با اختلافات و تَفْرَقه‌هایی که در میان آنان بوده و هست، همگان به مقام نبوت
آن یگانه مُعْتَرف، و در برابر آنچه آورده خاضع و خاکسار هستند.

اختلاف در اثبات است، یعنی: هر فرقه‌ای مدعی است که آنچه در نزد ماست او
آورده، (کل حزب بمالدیهم فرحون) از این جهت نزاع‌ها و جدال‌ها برخاسته، و
هر گروهی برای اثبات آنچه در نزد اوست، و ابطال خلافش استدلال می‌نماید.

و شکی نیست که اختلاف را رسول اکرم(صلی الله علیه و آله و سلم) نیاورده،
بلکه در میان امت پدید آمده، و بر هر مسلمانی لازم است آنچه را رسول خدا
نیاورده از ساحت دین دور نماید، و در تحقیق آنچه آورده بیش از همه چیز و
پیش از همه چیز بکوشد، تا در ادعایش که او رسول خداست و هر چه آورده حق
است صادق و استوار باشد.

پس باید حقیقتی که پیمبر آورده، و هر کسی سربسته به آن معترف است، و در
پس تعصبات اختلاف کنندگان نهان گشته، منصفانه و بی‌غرضانه جستجو کرد، و
اختلاف را رها نمود.

\subsection*{سخن هشتم}

ندای وحدت ندای الهی است، و ندای اختلاف ندای شیطانی است.

يَا أَيُّهَا الَّذِينَ آمَنُوا اتَّقُوا اللَّهَ حَقَّ تُقَاتِهِ وَلَا تَمُوتُنَّ إِلَّا وَأَنتُم مُّسْلِمُونَ -
وَاعْتَصِمُوا بِحَبْلِ اللَّهِ جَمِيعًا وَلَا تَفَرَّقُوا وَاذْكُرُوا نِعْمَتَ اللَّهِ عَلَيْكُمْ إِذْ كُنتُمْ
أَعْدَاءً فَأَلَّفَ بَيْنَ قُلُوبِكُمْ فَأَصْبَحْتُم بِنِعْمَتِهِ إِخْوَانًا وَكُنتُمْ عَلَىٰ شَفَا حُفْرَةٍ مِّنَ
النَّارِ فَأَنقَذَكُم مِّنْهَا كَذَٰلِكَ يُبَيِّنُ اللَّهُ لَكُمْ آيَاتِهِ لَعَلَّكُمْ تَهْتَدُونَ - وَلْتَكُن مِّنكُمْ
أُمَّةٌ يَدْعُونَ إِلَى الْخَيْرِ وَيَأْمُرُونَ بِالْمَعْرُوفِ وَيَنْهَوْنَ عَنِ الْمُنكَرِ وَأُولَٰئِكَ هُمُ
الْمُفْلِحُونَ - وَلَا تَكُونُوا كَالَّذِينَ تَفَرَّقُوا وَاخْتَلَفُوا مِن بَعْدِ مَا جَاءَهُمُ الْبَيِّنَاتُ
وَأُولَٰئِكَ لَهُمْ عَذَابٌ عَظِيمٌ - سوره آل عمران -۳- آیه ۱۰۲-۱۰۵

ای مومنان بپرهیزید خدا را آنچنان که سزاوار است از اون پرهیز کردن و
البته نمیرید جز در حالی که مسلمان باشید - و همگی به ریسمان خدا چنگ
زنید و پراکنده نشوید، و به یاد آورید نعمت خدا را بر شما آنگاه که با هم
دشمن بودید، پس خدا میان دل‌های شما الفت و یگانگی قرار داد، و شما در
پرتگاه آتش دوزخ بودید، پس خدا شما را از آن آتش رهایی بخشید، خدا آنچنان
بیان می‌کند برای شما آیاتش را شاید هدایت شوید - و از شما گروهی باشند که
سوی خیر دعوت نمایند، و امر به معروف و نهی از منکر کنند، و آنان هستند
رستگاران - و نباشید مانند کسانی که پراکنده شدند و اختلاف کردند پس از
آنکه برای آنان بَیِّنات آمد، و آنان را عذابی است بزرگ.

\subsubsection*{بیان}

ریسمان خدا، یعنی: حقیقتی که بندگان به وسیله آن با خدا ارتباط پیدا
می‌کنند و راه او را می‌پیمایند، و از این آیات استفاده می‌شود که یگانه علت
ایجاد وحدت چنگ زدن همگان است به ریسمان خدا، و در احادیث بسیاری از رسول
گرامی(صلی الله علیه و آله و سلم) روایت شده که ریسمان الاهی که بایستی
مؤمنان به آن چنگ زنند، یکی کتاب خدا و یکی خاندان نبوت است، و این دو
وسیله هدایت، و مانع از ضلالت، و تا روز قیامت از هم جدائی ندارند.


إِنَّ الَّذِينَ يَكْتُمُونَ مَا أَنزَلَ اللَّهُ مِنَ الْكِتَابِ وَيَشْتَرُونَ بِهِ ثَمَنًا قَلِيلًا أُولَٰئِكَ مَا
يَأْكُلُونَ فِي بُطُونِهِمْ إِلَّا النَّارَ وَلَا يُكَلِّمُهُمُ اللَّهُ يَوْمَ الْقِيَامَةِ وَلَا يُزَكِّيهِمْ وَلَهُمْ
عَذَابٌ أَلِيمٌ -‏ أُولَٰئِكَ الَّذِينَ اشْتَرَوُا الضَّلَالَةَ بِالْهُدَىٰ وَالْعَذَابَ بِالْمَغْفِرَةِ فَمَا
أَصْبَرَهُمْ عَلَى النَّارِ -‏ ذَٰلِكَ بِأَنَّ اللَّهَ نَزَّلَ الْكِتَابَ بِالْحَقِّ وَإِنَّ الَّذِينَ اخْتَلَفُوا فِي
الْكِتَابِ لَفِي شِقَاقٍ بَعِيدٍ - سوره البقره -۲- آیه ۱۷۴-۱۷۶

کسانی که کتمان می‌کنند کتابی را که خدا نازل کرده، و آن را به بهای اندکی
می‌فروشند در باطنشان نمی‌خورند جز آتش، و در روز قیامت خدا با آنان سخن
نمی‌گوید و آنان را پاک نمی‌نماید، و برای آنان عذابی است دردناک - آنان
کسانی هستند که هدایت را دادند و ضلالت را خریدند و مَغْفرت را دادند و
عذاب را خریدند، پس چه شکیبا هستند بر آتش دوزخ - آن جریان برای آن است
که خدا کتاب را به حق نازل کرد، و کسانی که در کتاب اختلاف کردند در
مخالفت عَمیقی هستند.

\subsubsection*{بیان}

یعنی: با حق در مخالفت هستند، زیرا اگر همگی از حقی که خدا نازل کرده
پیروی می‌کردند در کتاب خدا اختلاف نمی‌کردند، و امت اسلام در اصل کتاب خدا
اختلاف ندارند، ولی اختلاف آنان در تفسیر و تأویل آیات کم نیست، زیرا
بیان کلام خدا را از قرین کتاب خدا نگرفتند، و البته فهم و درک افراد
مختلف است، و هر کس از روی درک و فهم خودش چیزی گفته، و آیات الاهی را
برای خود تفسیر نموده، و در حدیثی که شیعه و اهل سنت آن را روایت
کرده‌اند، رسول خد(صلی الله علیه و آله و سلم) فرمود: هر کس قرآن را به
نظر خود تفسیر کند جای خود را در آتش جهنم گیرد.

إِنَّ الَّذِينَ فَرَّقُوا دِينَهُمْ وَكَانُوا شِيَعًا لَّسْتَ مِنْهُمْ فِي شَيْءٍ إِنَّمَا أَمْرُهُمْ إِلَى اللَّهِ
ثُمَّ يُنَبِّئُهُم بِمَا كَانُوا يَفْعَلُونَ - سورةُ الْانعام -۶- آیه ۱۵۹

آنان که دینشان را پراکنده کردند و گروه‌هایی شدند هر یک پیرو کسی، تو در
هیچ جهتی از ایشان نیستی، سر انجام کارشان بس با خداست، سپس آن‌ها را به
آنچه کرده‌اند آگاه می‌نماید.

فَأَقِمْ وَجْهَكَ لِلدِّينِ حَنِيفًا فِطْرَتَ اللَّهِ الَّتِي فَطَرَ النَّاسَ عَلَيْهَا لَا تَبْدِيلَ لِخَلْقِ
اللَّهِ ذَٰلِكَ الدِّينُ الْقَيِّمُ وَلَٰكِنَّ أَكْثَرَ النَّاسِ لَا يَعْلَمُونَ -‏ مُنِيبِينَ إِلَيْهِ وَاتَّقُوهُ
وَأَقِيمُوا الصَّلَاةَ وَلَا تَكُونُوا مِنَ الْمُشْرِكِينَ ‎- مِنَ الَّذِينَ فَرَّقُوا دِينَهُمْ وَكَانُوا
شِيَعًا كُلُّ حِزْبٍ بِمَا لَدَيْهِمْ فَرِحُونَ - سورةُ الرّوم -۳۰- آیه ۳۰-۳۲.

روی خود را بهر دین با میل دل کن استوار، فطرت الهی را بنگر، آن فطرتی که
مردم را بر آن آفریده، برای آفرینش خدا تبدیلی نیست، آن است دین استوار،
ولکن بیشتر مردم نمی‌دانند - توجه کنندگان به سوی او بدانید و از او
بپرهیزید و نماز را به پا دارید و از مشرکان نباشید - از آن کسانی که
دینشان را پراکنده کردند، و گروه‌هایی شدند، هر یک پیرو کسی هر حزبی به
آنچه نزدشان می‌باشد شادمانند.

وَأَنَّ هَٰذَا صِرَاطِي مُسْتَقِيمًا فَاتَّبِعُوهُ وَلَا تَتَّبِعُوا السُّبُلَ فَتَفَرَّقَ بِكُمْ عَن سَبِيلِهِ ذَٰلِكُمْ
وَصَّاكُم بِهِ لَعَلَّكُمْ تَتَّقُونَ - سورة الانعام -۶- آیه ۱۵۳.

و این است راه من که راست است، پس پیروی کنیدش، و پیروی نکنید راه‌های
مختلف را که شما را از راه خدا جدا می‌کند، خدا شما را به آن راه سفارش
کرده شاید پرهیزکار شوید.

شَرَعَ لَكُم مِّنَ الدِّينِ مَا وَصَّىٰ بِهِ نُوحًا وَالَّذِي أَوْحَيْنَا إِلَيْكَ وَمَا وَصَّيْنَا بِهِ إِبْرَاهِيمَ
وَمُوسَىٰ وَعِيسَىٰ أَنْ أَقِيمُوا الدِّينَ وَلَا تَتَفَرَّقُوا فِيهِ كَبُرَ عَلَى الْمُشْرِكِينَ مَا تَدْعُوهُمْ
إِلَيْهِ اللَّهُ يَجْتَبِي إِلَيْهِ مَن يَشَاءُ وَيَهْدِي إِلَيْهِ مَن يُنِيبُ - سورة الشّوری -۴۲- آیه
۱۳.

راهی از دین قرار داد برای شما آنچه را به نوح توصیه نمود، و آنچه سوی تو
وحی کرد و آنچه را به ابراهیم و موسی و عیسی توصیه فرمود، که به یاد آرید
دین را، و در آن پراکنده نشوید، ناگوار است بر مشرکان آن چیزی که سوی آن
می‌خوانی آنان را، خدا هر کسی را خواهد به دینش اختصاص می‌دهد، و هر کس به
خدا توجه داشته باشد، سوی دینش هدایت می‌نماید.

\subsubsection*{بیان}

این آیات بندگان را به وحدت دعوت می‌کند، و از تفرقه و اختلاف بر حذر
می‌نماید، و وحدت امت در زمان رسول اکرم(صلی الله علیه و آله و سلم)
برقرار بوده، و البته خدای متعال برای حفظ آن پس از رحلت رسولش چاره‌ای
نموده، ولی امت حفظ نکردند، و گرفتار زیان اختلاف شدند.

\subsection*{سخن نهم}

و زیان اختلاف در آخرت است، و جبران پذیر نیست، و رسول خدا(صلی الله علیه
و آله و سلم) در حدیثی اخطار فرموده:

قالَ رَسُولُ اَللَّهِ صَلَّى اَللَّهُ عَلَيْهِ وَ آلِهِ و سَلَّمَ: اِفْتَرَقَتِ اَلْيَهُودُ إِحْدَى وَ سَبْعِينَ
فِرْقَةً، سَبْعُونَ مِنْهَا فِي اَلنَّارِ وَ وَاحِدَةٌ فِي اَلْجَنَّةِ، وَ اِفْتَرَقَتِ اَلنَّصَارَى اِثْنَتَيْنِ وَ
سَبْعِينَ فِرْقَةً، إِحْدَى وَ سَبْعُونَ فِرْقَةً فِي اَلنَّارِ وَ وَاحِدَةٌ فِي الْجَنَّةِ، وَ أُمَّتِي سَتَفْتَرِقُ
ثَلاَثاً وَ سَبْعِينَ فِرْقَةً، ثِنَتَانِ وَ سَبْعُونَ فِرْقَةً فِي اَلنَّارِ وَ وَاحِدَةٌ فِي الْجَنَّةِ.


رسول خدا(صلی الله علیه و آله و سلم) فرمود: یهودیان هفتاد و یک فرقه
شدند، هفتاد فرقه در آتش، و یک فرقه در بهشت هستند، و نصرانیان هفتاد و
دو فرقه شدند، هفتاد و یک فرقه در آتش و یک فرقه در بهشت هستند، و امت من
هفتاد و سه فرقه می‌شوند، هفتاد و دو فرقه در آتش و یک فرقه در بهشت
هستند.

\subsubsection*{بیان}

این حدیث مشهور است، شیعه و اهل سنت آن را با تفاوت کلمات روایت کرده‌اند،
و بسیاری از دانشمندان در شرح این حدیث کتاب‌ها تألیف کرده‌اند، و نام و
نشان هفتاد و سه فرقه را بیان کرده‌اند، و این اخطاری است از رسول خدا(صلی
الله علیه و آله و سلم) پس بر هر مسلمانی لازم است که تحقیق کند و یقین
نماید که از آن یک فرقه بهشتی است، وگرنه خطر بسیار بزرگ است، و قابل
جبران نیست، و این تحقیق راهی است برای ایجاد وحدت.
