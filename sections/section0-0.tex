سپاس خدا راست که انسان آفرید، و برای تکامل عرفان به این خاکدانش کشید،
و درود بی‌پایان، بر سرور پیامبران و خاندان پاکش که مکتب عالی آموزش
انسانی را گشودند، و جهان را با نشر تعالیم و معارف الاهی بوستان معرفت
نمودند.

این کتاب قطره‌ای از دریای بی‌کران فضائل خاندان نبوّت است، به اهل ایمان و
محبّت اهداء می‌شود، و امید از خدای منان که عنایت فرماید، و ما را در دو
جهان با آنان محشور بدارد، اکنون سخنی چند به عنوان پیش‌داشت مذکور
می‌داریم.

\subsection*{سخن اول}

اصل هر فضیلتی دانش است، و هر فضیلتی در ظرفِ جهل وارونه گردد، شجاعِ جاهل،
مُتَهَوِّر است، خود و دیگران را بیهوده به هلاکت اندازد، سخیِّ جاهل، مُسرِف است،
مال را بی‌جا تلف نماید، حَلیمِ جاهل مُنْظَلِم است، بی‌جهت زیر بار ستم رود،
عابد جاهل، مُبْدِع است، عباداتی بر خلاف سُنّت آورد، زاهد جاهل، مُهْمل کار
است، بسا زهدها کند که ترک وظیفه باشد، واعظ جاهل مُفْسِدْ است، بسا سخنان
گوید که مردم را مُنْحَرِف نماید، قاضی جاهل ظالم است، بسا داوری‌ها کند که حق
مظلوم ضایع شود، شرمسار جاهل بی‌جا شرم نماید، و خود را از مزایایی محروم
سازد، راستگوی جاهل فتنه انگیز است، بسا راست‌ها گوید و فتنه‌ها از آن
برخیزد، ووو.

پس علم است که هر رذیلتی را جلوگیر، و هر فضیلتی را به جا استوار نماید،
و از این روی دانسته شود که مِلاکِ برتری هر فردی، بر فردی و هر اجتماعی بر
اجتماعی دانش است.

ولی به شرط تَقوی، و دانشمندِ بی‌تقوی از نادان پست‌تر است، زیرا برای خود و
جامعه زیان‌‌بار است، ولی نادان خود را از منافعی محروم ساخته.

خدای متعال گرچه دانش و دانشمند را در آیاتی ستوده، ولی گرامی بودن نزد
خودش را به شرط تقوی گوشزد نموده، إِنَّ أَكْرَمَكُمْ عِنْدَ اللَّهِ أَتْقَاكُمْ: همانا
گرامی‌ترین شما نزد خدا پرهیزکارترین شماست.

خلاصه سخن دو اصل مُثْبَت و مَنفی برای تکامل بشری لازم است: علم و تقوی، و
هر یک به تنهایی ثمر بخش نیست.

\subsection*{سخن دوم}

در این کتاب احادیثی از رسول اکرم(صَلَّی اللهُ عَلَیْه و آله و سلّم) در چهل بخش
درباره علم خاندان نُبُوَّت(صَلَواتُ الله علیهم) آورده‌ام، و بخش چهلم، سخنان
برخی از صحابه و تابعین و دانشمندان است، و آن‌ها گرچه وانمود سخنان پیمبر
است، لکن گواهی است، که امت همگی بفضل علمی آنان اعتراف دارند، و در این
باره کسی بر سر انکار نیست، و به شهادت احادیثی از رسول خدا(صلی الله
علیه و آله و سلم) فضائل آنان بی‌نهایت است، برخی از آن احادیث:

عَنِ اِبْنِ عَبَّاسٍ رَضِیَ اللهُ عَنْهُ قَالَ: قَالَ رَسُولُ اَللَّهِ صَلَّى اَللَّهُ عَلَيْهِ وَ آلِهِ: لَوْ أَنَّ
اَلْغِيَاضَ أَقْلاَمٌ وَ اَلْبَحْرَ مِدَادٌ وَ اَلْجِنَّ حُسَّابٌ وَ اَلْإِنْسَ كُتَّابٌ مَا أَحْصَوْا فَضَائِلَ عَلِيِّ
بْنِ أَبِي طَالِبٍ عَلَیْهِ السَّلامُ.

ابن عباس(رَضِیَ اللهُ عَنْهُ) گفت: رسول خدا(صَلَّى اَللَّهُ عَلَيْهِ وَ آلِهِ وَ سَلَّمَ) فرمود:
اگر درختان قلم، و دریاها مداد، و پریان حسابگر، و آدمیان نویسنده باشند،
فضائل علی بن ابی طالب را نتوان به حساب آورند.

عَنْ اَمیرِالْمُوْمِنینَ عَلَیْهِ السَّلامُ قالَ: قَالَ رَسُولُ اَللَّهِ صَلَّى اَللَّهُ عَلَيْهِ وَ آلِهِ: إِنَّ
اَللَّهَ تَعَالَى جَعَلَ لِأَخِي عَلِيٍّ فَضَائِلَ لاَ تُحْصَى كَثْرَةً فَمَنْ ذَكَرَ فَضِيلَةً مِنْ فَضَائِلِهِ
مُقِرّاً بِهَا غَفَرَ اَللَّهُ لَهُ مَا تَقَدَّمَ مِنْ ذَنْبِهِ وَ مَا تَأَخَّرَ وَ مَنْ كَتَبَ فَضِيلَةً مِنْ
فَضَائِلِهِ لَمْ تَزَلِ اَلْمَلاَئِكَةُ تَسْتَغْفِرُ لَهُ مَا بَقِيَ لِتِلْكَ اَلْكِتَابَةِ رَسْمٌ وَ مَنِ اِسْتَمَعَ
إِلَى فَضِيلَةٍ مِنْ فَضَائِلِهِ غَفَرَ اَللَّهُ لَهُ اَلذُّنُوبَ اَلَّتِي اِكْتَسَبَهَا بِالاِسْتِمَاعِ وَ مَنْ نَظَرَ
إِلَى كِتَابٍ مِنْ فَضَائِلِهِ غَفَرَ اَللَّهُ لَهُ اَلذُّنُوبَ اَلَّتِي اِكْتَسَبَهَا بِالنَّظَرِ ثُمَّ قَالَ اَلنَّظَرُ
إِلَى أَخِي عَلِيِّ بْنِ أَبِي طَالِبٍ عِبَادَةٌ وَ ذِكْرُهُ عِبَادَةٌ وَ لاَ يَقْبَلُ اَللَّهُ إِيمَانَ عَبْدٍ
إِلاَّ بِوَلاَيَتِهِ وَ اَلْبَرَاءَةِ مِنْ أَعْدَائِهِ.

امیرالمومنین(علیه السلام) گفت: رسول خدا(صَلَّى اَللَّهُ عَلَيْهِ وَ آلِهِ وَ سَلَّمَ)
فرمود: خدا برای برادرم علی فضائِلی قرار داده که از بسیاری به حساب
نیاید، پس هر کس فضیلتی از فضائل او را یاد کند و به آن اعتراف داشته
باشد، خدا گناهان گذشته و آینده‌اش را بیامرزد.

و هر کس فضیلتی از فضائل او را بنویسد تا وقتی از آن نوشته اثری باقی
است، فرشتگان همواره برای او از خدا آمرزش می‌طلبند.

و هر کس فضیلتی از فضائل علی را به گوش فراگیرد، خدا گناهانی را که با
شنیدن مرتکب شده بخشش نماید.

و هر کس در کتابی از فضائل علی بنگرد، خدا گناهانی را که با دیدن مرتکب
شده آمرزش نماید.


سپس رسول خدا(صَلَّى اَللَّهُ عَلَيْهِ وَ آلِهِ وَ سَلَّمَ) فرمود: نظر کردن به برادرم علی
بن ابی طالب عبادت است، و ذکر علی عبادت است، و خدا ایمان بنده‌ای را
نپذیرد جز به ولایت علی و بیزاری از دشمنان علی.

این دو حدیث را روایت کرده:

\begin{enumerate}
\item

حافظ اَبوالْمُؤَیَّد مُوفَّق بن اَحمد بن محمد بَکْری مَکّی حَنفی مَذْهب، معروف به خطیب
خوارَزمی و اَخْطَبِ خوارزم که در سال ۵۶۸ هجری قمری بدرود جهان گفته، در آغاز
کتابش به نام (مَناقب خوارزمی).

\item

اَبوالمَجامِع ابراهیم بن محمد بن الْمُؤَیَّد بن حَمّویَه جُوَیْنی خراسانی حنفی مذهب
معروف به حَمُّویی که در سال ۷۳۰ بدرود جهان گفته، در آغاز کتابش (فَرائِد
السِّمْطَیْنِ).

\item

 سِبْطِ اِبْنِ‌ الْجَوْزی‌‌، یوسف‌ بن‌ قُزُغْلى‌ بن عبدالله بغدادی، حنفی مذهب، که در سال
 ۶۵۴ بدرود جهان گفته، در آغاز کتاب (تَذْکِرَةُ الْخَواصّ) صفحه ۱۳، حدیث اول را
 آورده.

\item
  
ابو عبدالله محمد بن یوسف محمد نَوْفَلی كَنْجی شافعی مذهب كه در سال ۶۵۸ در
مسجد جامع شام بدست گروهی آشوب‌گر مقتول گشته، در كتاب (كِفایَةُ الطّالِب) باب
شصت و دوم، صفحه ۱۲۳ و ۱۲۴.

\end{enumerate}
  
