سپاس خدا راست که انسان آفرید، و برای تکامل عرفان به این خاکدانش کشید،
و درود بی‌پایان، بر سرور پیامبران و خاندان پاکش که مکتب عالی آموزش
انسانی را گشودند، و جهان را با نشر تعالیم و معارف الاهی بوستان معرفت
نمودند.

این کتاب قطره‌ای از دریای بی‌کران فضائل خاندان نبوّت است، به اهل ایمان و
محبّت اهداء می‌شود، و امید از خدای منان که عنایت فرماید، و ما را در دو
جهان با آنان محشور بدارد، اکنون سخنی چند به عنوان پیش‌داشت مذکور
می‌داریم.

\subsection{سخن اول}

اصل هر فضیلتی دانش است، و هر فضیلتی در ظرفِ جهل وارونه گردد، شجاعِ جاهل،
مُتَهَوِّر است، خود و دیگران را بیهوده به هلاکت اندازد، سخیِّ جاهل، مُسرِف است،
مال را بی‌جا تلف نماید، حَلیمِ جاهل مُنْظَلِم است، بی‌جهت زیر بار ستم رود،
عابد جاهل، مُبْدِع است، عباداتی بر خلاف سُنّت آورد، زاهد جاهل، مُهْمل کار
است، بسا زهدها کند که ترک وظیفه باشد، واعظ جاهل مُفْسِدْ است، بسا سخنان
گوید که مردم را مُنْحَرِف نماید، قاضی جاهل ظالم است، بسا داوری‌ها کند که حق
مظلوم ضایع شود، شرمسار جاهل بی‌جا شرم نماید، و خود را از مزایایی محروم
سازد، راستگوی جاهل فتنه انگیز است، بسا راست‌ها گوید و فتنه‌ها از آن
برخیزد، ووو.
